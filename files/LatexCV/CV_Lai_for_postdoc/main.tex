% --- LaTeX CV Template - S. Venkatraman ---

% --- Set document class and font size ---

\documentclass[letterpaper,11pt]{article}

% --- Package imports ---

\usepackage{hyperref, enumitem, longtable, amsmath, array}

% --- Page layout settings ---

% Set page margins
\usepackage[left=0.7in, right=0.8in, bottom=.8in, top=0.8in, headsep=0in, footskip=.2in]{geometry}

% Set line spacing
\renewcommand{\baselinestretch}{1.2}

% --- Page formatting settings ---

% Set link colors
\usepackage[dvipsnames]{xcolor}
\hypersetup{colorlinks=true, linkcolor=MidnightBlue, urlcolor=MidnightBlue}

% Set font to Libertine, including math support
\usepackage{libertine}
\usepackage[libertine]{newtxmath}

% Remove page numbering
%\pagenumbering{gobble}

% page numbering
\usepackage{fancyhdr}  % 导入 fancyhdr 包,用于自定义页眉和页脚
\fancyhf{}  % 清除页眉和页脚的默认设置
\fancyhead[L]{Ph.D. Zhijian Lai}  % 设置页眉左侧的内容为 "PhD Zhijian Lai"
\fancyhead[R]{University of Tsukuba}  % 设置页眉右侧的内容为 "University of Tsukuba"
\fancyfoot[C]{\thepage/\pageref{LastPage}}  % 设置页脚中心的内容为 "当前页码/总页数"
\renewcommand{\headrulewidth}{0.4pt}  % 设置页眉下的横线宽度为 0.4pt
\renewcommand{\footrulewidth}{0.4pt}  % 设置页脚上的横线宽度为 0.4pt

\fancypagestyle{plain}{  % 定义一个新的页面样式 'plain'
	\fancyhead{}  % 清除页眉的默认设置
	\renewcommand{\headrulewidth}{0pt}  % 将 'plain' 样式的页眉下的横线宽度设置为 0,即去除横线
}

\pagestyle{fancy}  % 设置默认页面样式为 'fancy'



% Define font size and color for section headings
\newcommand{\headingfont}{\Large\color{OliveGreen}}

% --- CV section settings ---

% Note: each section of this table (Education, Awards, Publications etc.) is 
% stored in a two-column table. The left-hand column is narrow (1 inch) and is 
% meant to store dates. The right-hand column is wide (5.2 inches) and stores 
% the main text.  

% Sections in which each entry might have multiple lines 
% (e.g., Education) are stored in a 'SectionTable' environment). Sections in 
% which each entry might just have one line are stored in a 'SectionTableSingleSpace'
% environment. The only difference between the two environments is the line 
% spacing between each entry. Both environments take one argument, which is the
% title of the section. See main document for how these environments are used.

% Define settings for left-hand column in which dates are printed
\newcolumntype{R}{>{\raggedleft}p{1in}}
%这段 LaTeX 代码定义了一个新的列类型 `R`,它用于表格中的列格式设置。
%
%下面是对这段代码的解释:
%
%- `\newcolumntype{R}{>{\raggedleft}p{1in}}`:这是 LaTeX 中定义新的列类型的命令。`R` 是新列类型的名称,`>{\raggedleft}p{1in}` 是这个新列类型的定义。
%
%`>{\raggedleft}p{1in}` 可以被分解为两部分:
%
%1. `>{\raggedleft}`:这个部分使用了 `array` 包提供的 `>` 命令,该命令用于在每个单元格开始之前插入一些代码。在这里,它插入了 `\raggedleft` 命令,该命令会使得列中的文本右对齐。
%
%2. `p{1in}`:这个部分定义了列的宽度。`p{1in}` 定义了一个具有固定宽度的段落列,宽度为 `1in`(1 英寸)。
%
%所以,这段代码定义的 `R` 列类型将会创建一个宽度为 `1in` 的列,其中的文本右对齐。


% Define 'SectionTable' environment
\newenvironment{SectionTable}[1]{
	\renewcommand*{\arraystretch}{1.7}
	\setlength{\tabcolsep}{10pt}
	\begin{longtable}{Rp{5.2in}} & #1 \\ % `R` 和 `p{5.2in}` 是两个列的格式
	}
	{
	\end{longtable}\vspace{-.3cm}
}

%这段 LaTeX 代码定义了一个新的环境 `SectionTable`,这个环境主要用于创建一个特殊样式的长表格(`longtable`)。
%
%下面是这段代码的逐行解释:
%
%- `\newenvironment{SectionTable}[1]{...}{...}`:这是 LaTeX 中定义新环境的命令。`SectionTable` 是新环境的名称,`[1]` 表示这个环境接受一个参数。
%
%环境的定义有两部分:
%
%1. 第一部分 `{...}` 是在进入环境时执行的代码:
%- `\renewcommand*{\arraystretch}{1.7}`:这行代码改变了表格的行高。`arraystretch` 是一个比例因子,原始的行高会乘以这个比例因子。默认的值是 `1`,这里将它设置为 `1.7`,所以新的行高会是原始行高的 `1.7` 倍。
%- `\setlength{\tabcolsep}{10pt}`:这行代码设置了表格中的列间距(也就是列与列之间的空白距离)。这里将它设置为 `10pt`。
%- `\begin{longtable}{Rp{5.2in}} & #1 \\}`:这行代码开始了一个 `longtable` 环境,并且指定了列格式。`R` 和 `p{5.2in}` 是两个列的格式,`R` 可能是一个自定义的列类型。`#1` 是传递给 `SectionTable` 环境的参数,它将作为表格的第一行的内容。最后的 `\\` 是换行命令,表示结束这一行。
%
%2. 第二部分 `{...}` 是在离开环境时执行的代码:
%- `\end{longtable}\vspace{-.3cm}`:这行代码结束了 `longtable` 环境,并且添加了一个负的垂直空间 `-0.3cm`。负的垂直空间实际上会减少后面内容的前导空白。
%
%总的来说,`SectionTable` 环境是为了创建一个样式特定的长表格。这个表格接受一个参数作为第一行的内容,列格式是 `R` 和 `p{5.2in}`,行高是原始行高的 `1.7` 倍,列间距是 `10pt`,并且在表格结束后减少 `0.3cm` 的前导空白。





% Define 'SectionTableSingleSpace' environment
\newenvironment{SectionTableSingleSpace}[1]{
	\renewcommand*{\arraystretch}{1.2}
	\setlength{\tabcolsep}{10pt}
	\begin{longtable}{Rp{5.2in}} & #1 \\[0.6em]
	}
	{
\end{longtable}\vspace{-.3cm}
}

% --- Document starts here ---

\begin{document}
	

\thispagestyle{plain}  % 设置第一页的样式为 'plain'

% --- Name and contact information ---


\begin{SectionTable}{\Huge Zhijian Lai} 
& 
%s2130117@s.tsukuba.ac.jp $\;\boldsymbol{\cdot}\;$ www.yourwebsite.com $\;\boldsymbol{\cdot}\;$ 314-159-2654 \newline
%Citizenship: Country
{Email}: s2130117@s.tsukuba.ac.jp \newline
{Office}: 3E310, 1-1-1 Tennodai, Tsukuba, Ibaraki, 305-8577, Japan \newline
{Homepage}: \footnotesize\url{https://galvinlai.github.io/}
\end{SectionTable}

% --- Section: Research interests ---

\begin{SectionTable}{\headingfont Research interests}
& Mathematical Optimization, Riemannian Optimization, Machine Learning, Deep Learning, Quantum Computing
\end{SectionTable}

% --- Section: Education ---

\begin{SectionTable}{\headingfont Education}
2021 -- Present & 
\textbf{University of Tsukuba} -- Tsukuba, Japan\newline
Ph.D. in Policy and Planning Sciences \newline 
Supervisor:
\href{https://infoshako.sk.tsukuba.ac.jp/~yoshise/}{Prof. Akiko Yoshise}\\

2019 -- 2021 & 
\textbf{University of Tsukuba} -- Tsukuba, Japan\newline
Master of Science (M.S.) in Policy and Planning Sciences \newline 
Supervisor:
\href{https://infoshako.sk.tsukuba.ac.jp/~yoshise/}{Prof. Akiko Yoshise}\\

2013 -- 2017 & 
\textbf{Dongbei University of Finance and Economics} -- Dalian, China \newline
Bachelor of Management (B.Mgmt.)\\ 

%Mentors: Professors E, F. \textit{GPA: X.YZ}. \\

% --- Un-comment the next few lines if you want to include some courses you've taken ---

%& \textbf{Selected coursework}
%\begin{itemize}[itemsep=0pt, leftmargin=*]
%\item \textit{Statistics}: Asymptotic statistics, Mathematical statistics, Functional data analysis, High-dimensional statistics, Information theory
%\item \textit{Mathematics}: Measure theory, Functional analysis, Measure-theoretic probability with martingales
%\end{itemize}

\end{SectionTable}


\begin{SectionTable}{\headingfont Grants}
	2021 -- Present &
	Research fellowship of \textit{Support for Pioneering Research Initiated by the Next Generation} (SPRING), Japan Science and Technology Agency
\end{SectionTable}


%% --- Section: Awards, scholarships, etc ---
%
%\begin{SectionTableSingleSpace}{\headingfont Honors and scholarships}
%2021 & 
%Award 1 (Organization that gave you the award)\newline
%\textit{Maybe this award needs a short description}. \\
%
%2020 &
%Award 2 (Organization that gave you the award; \href{https://en.wikibooks.org/wiki/LaTeX/Hyperlinks}{link if you want}) \\
%
%2019 &
%Award 3 (Organization that gave you the award) \\
%
%2018 &
%Award 4 (Organization that gave you the award) \\
%
%2017 &
%Award 5 (Organization that gave you the award)
%%\end{SectionTableSingleSpace}

% --- Section: Publications ---

\begin{SectionTable}{\headingfont Publications and Preprints} 
2022 & 
\textbf{Riemannian Interior Point Methods for Constrained Optimization on Manifolds} \newline
{Zhijian Lai}, Akiko Yoshise. \newline
\textit{arxiv.org/abs/2203.09762} (Under review). \\

2022 & 
\textbf{Completely Positive Factorization by a Riemannian Smoothing Method} \newline
{Zhijian Lai}, Akiko Yoshise. \newline
\textit{Computational Optimization and Applications}. \\
\end{SectionTable}


\begin{SectionTable}{\headingfont Working Papers} 
	2023 & 
	\textbf{CLAP: A Contrastive Learning Structure for App-usage Prediction} \newline
	Xin Yang, Zhijian Lai, Qian Wu, Maiko Shigeno. \\
	
	2023 & 
	\textbf{HGCL4REC: Hyperbolic Graph Contrastive Learning for Recommender System} \newline
	Xin Yang, Zhijian Lai, Qian Wu, Maiko Shigeno. \\
\end{SectionTable}







% --- Section: Research experience ---

\begin{SectionTable}{\headingfont Research experience}
Month Year -- Present &
\textbf{Title of project or lab where research was conducted} \newline
Mentors: Professor A (University). \newline
Description of your work. Summary of findings available \href{https://en.wikibooks.org/wiki/LaTeX/Hyperlinks}{here}. Sed dolor lacus, imperdiet non, ornare non, commodo eu, neque. Integer pretium semper justo. \\

Month Year -- Month Year &
\textbf{Title of project or lab where research was conducted} \newline
Mentors: Professor B (University). \newline
Description of your work. Summary of findings available \href{https://en.wikibooks.org/wiki/LaTeX/Hyperlinks}{here}. Sed dolor lacus, imperdiet non, ornare non, commodo eu, neque. Integer pretium semper justo. \\
\end{SectionTable}

% --- Section: Teaching experience ---

\begin{SectionTable}{\headingfont Teaching experience}
Fall 2020 & 
\textbf{Teaching assistant, STAT 123: Course name here (University)} \newline
Topics and description of your responsibilities. Aliquam volutpat est vel massa. Sed dolor lacus, imperdiet non, ornare non, commodo eu, neque. \newline
\textit{Average student rating: X/5.} \\

Spring 2020 & 
\textbf{Teaching assistant, MATH 234: Course name here (University)} \newline
Topics and description of your responsibilities. Aliquam volutpat est vel massa. Sed dolor lacus, imperdiet non, ornare non, commodo eu, neque. \newline
\textit{Average student rating: X/5.}
\end{SectionTable}

%% --- Section: Industry experience ---
%
%\begin{SectionTable}{\headingfont Industry experience}
%Summer 2020 &
%\textbf{Name of company (Title of job or internship)} -- City, State \newline
%Description of your responsibilities. Integer pretium semper justo. Proin risus. Nullam id quam. Nam neque. Phasellus at purus et lib ero lacinia dictum.  \\
%
%Summer 2019 &
%\textbf{Name of company (Title of job or internship)} -- City, State \newline
%Description of your responsibilities. Integer pretium semper justo. Proin risus. Nullam id quam. Nam neque. Phasellus at purus et lib ero lacinia dictum.  \\
%
%Summer 2018 &
%\textbf{Name of company (Title of job or internship)} -- City, State \newline
%Description of your responsibilities. Integer pretium semper justo. Proin risus. Nullam id quam. Nam neque. Phasellus at purus et lib ero lacinia dictum.  \\
%\end{SectionTable}

% --- Section: Talks and tutorials ---


\begin{SectionTable}{\headingfont Conference Talks}
	Aug. 2023 & ICIAM 2023 \newline
	\textit{Riemannian Interior Point Methods for Constrained Optimization on Manifolds}, Tokyo. \\
	
	June 2023 & SIAM OP23 \newline 
	\textit{Interior Point Methods for Nonlinear Optimization on Riemannian Manifolds}, Seattle. \\
	
%	May 2023 &	RAOTA: Gathering of Young Researchers for the Future 2023 \newline 
%	\textit{Riemannian Interior Point Methods for Constrained Optimization on Manifolds}, Tsukuba, Japan. \\
	
	Mar. 2023 & The 2023 spring national conference of Operations Research Society of Japan \newline 
	\textit{Riemannian Interior Point Methods for Constrained Optimization on Manifolds}, Tokyo, Japan. \\
	
	Dec. 2022 & International Workshop on Continuous Optimization \newline 
	\textit{Riemannian Interior Point Methods for Constrained Optimization on Manifolds}, Tokyo (virtual). \\
	
	Sep. 2022 & The 2022 autumn national conference of Operations Research Society of Japan \newline
	\textit{On the Global Convergence of Riemannian Interior Point Method}, Niigata (virtual), Japan. \\
	
	Sep. 2022 & The Japan Society for Industrial and Applied Mathematics 2022 annual meeting \newline
	\textit{On the Global Convergence of Riemannian Interior Point Method}, Sapporo, Japan. \\
	
	Mar. 2022 & The 2022 spring national conference of Operations Research Society of Japan \newline
	\textit{Superlinear and Quadratic Convergence of Riemannian Interior Point Methods}, Gunma (virtual), Japan. \\
	
	July 2021 &	SIAM OP21 \newline 
	\textit{Completely Positive Factorization via Orthogonality Constrained Problem}, Hong Kong (virtual). \\
	
	Aug. 2021 & Meeting 2021 of Kyoto University Research Institute for Mathematical Sciences \newline
	\textit{Application of Smoothing Methods for Completely Positive Matrices via Orthogonality Constrained Problem}, Kyoto (virtual), Japan. \\
	
	Mar. 2021 & The 2021 spring national conference of Operations Research Society of Japan \newline
	\textit{Completely Positive Factorization via Orthogonality Constrained Problem}, Tokyo (virtual), Japan. \\
	
	Aug. 2020 & Meeting 2020 of Kyoto University Research Institute for Mathematical Sciences \newline
	\textit{A New Approach to the Recognition Problem of Completely Positive Matrices}, Kyoto (virtual), Japan. \\
	
\end{SectionTable}

%\begin{SectionTable}{\headingfont Talks and tutorials}
%Month Year &
%Title of your most recent presentation \newline
%\textit{Name of conference, workshop, seminar, etc., or a description} \\
%
%Month Year &
%Title of your second most recent presentation \newline
%\textit{Name of conference, workshop, seminar, etc., or a description} \\
%
%Month Year &
%Title of your third most recent presentation \newline
%\textit{Name of conference, workshop, seminar, etc., or a description} \\
%\end{SectionTable}


% --- Section: Mentorship and service ---

\begin{SectionTable}{\headingfont Mentorship and service}
Month Year -- Present &
\textbf{Title of organization you are in (Name of your role)} \newline
Description of your responsibilities. Integer pretium semper justo. Proin risus. Nullam id quam. Nam neque. Phasellus at purus et lib ero lacinia dictum. \\

Month Year -- Month Year &
\textbf{Title of organization you were in (Name of your role)} \newline
Description of your responsibilities. Integer pretium semper justo. Proin risus. Nullam id quam. Nam neque. Phasellus at purus et lib ero lacinia dictum. \\
\end{SectionTable}

% --- Section: Professional society memberships ---

\begin{SectionTable}{\headingfont Professional memberships}
Year -- Present &
Name of professional society \newline
\textit{Short description or conferences you attended.} \\

Year -- Present &
Name of professional society \newline
\textit{Short description or conferences you attended.} \\
\end{SectionTable}

\begin{SectionTable}{\headingfont Technical skills}
& \textbf{Programming languages} \newline
Proficient in: language 1, language 2, language 3 \newline
Familiar with: language 4, language 5 \\

& \textbf{Software} \newline
\LaTeX, Git, another piece of software \\

& \textbf{Languages} \newline
English (fluent), Another language (advanced)
\end{SectionTable}

% --- Section: Other interests/hobbies ---

\begin{SectionTable}{\headingfont Other interests}
& Some of your hobbies, etc.
\end{SectionTable}

% --- End of CV! ---

\label{LastPage}  % 在文档的最后添加一个标签,以便于引用总页数。

\end{document}





