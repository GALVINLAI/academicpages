% --- LaTeX CV Template - S. Venkatraman ---

% --- Set document class and font size ---

\documentclass[12pt]{article}

% --- Package imports ---

\usepackage{hyperref, enumitem, longtable, amsmath, array}

% --- Page layout settings ---

% Set page margins
\usepackage[left=0.7in, right=0.8in, bottom=.8in, top=0.8in, headsep=0in, footskip=.2in]{geometry}

% Set line spacing
\renewcommand{\baselinestretch}{1.2}

% --- Page formatting settings ---

% Set link colors
\usepackage[dvipsnames]{xcolor}
\hypersetup{colorlinks=true, linkcolor=MidnightBlue, urlcolor=MidnightBlue}

% Set font to Libertine, including math support
\usepackage{libertine}
\usepackage[libertine]{newtxmath}

% Remove page numbering
%\pagenumbering{gobble}

% page numbering
\usepackage{fancyhdr}  % 导入 fancyhdr 包,用于自定义页眉和页脚
\fancyhf{}  % 清除页眉和页脚的默认设置
\fancyhead[L]{Ph.D. Zhijian Lai}  % 设置页眉左侧的内容为 "PhD Zhijian Lai"
\fancyhead[R]{University of Tsukuba}  % 设置页眉右侧的内容为 "University of Tsukuba"
\fancyfoot[C]{\thepage/\pageref{LastPage}}  % 设置页脚中心的内容为 "当前页码/总页数"
\renewcommand{\headrulewidth}{0.4pt}  % 设置页眉下的横线宽度为 0.4pt
\renewcommand{\footrulewidth}{0.4pt}  % 设置页脚上的横线宽度为 0.4pt

\fancypagestyle{plain}{  % 定义一个新的页面样式 'plain'
	\renewcommand{\headrulewidth}{0pt}  % 将 'plain' 样式的页眉下的横线宽度设置为 0,即去除横线
}

\pagestyle{fancy}  % 设置默认页面样式为 'fancy'



% Define font size and color for section headings
\newcommand{\headingfont}{\Large\color{OliveGreen}}

% --- CV section settings ---

% Note: each section of this table (Education, Awards, Publications etc.) is 
% stored in a two-column table. The left-hand column is narrow (1 inch) and is 
% meant to store dates. The right-hand column is wide (5.2 inches) and stores 
% the main text.  Sections in which each entry might have multiple lines 
% (e.g., Education) are stored in a 'SectionTable' environment). Sections in 
% which each entry might just have one line are stored in a 'SectionTableSingleSpace'
% environment. The only difference between the two environments is the line 
% spacing between each entry. Both environments take one argument, which is the
% title of the section. See main document for how these environments are used.

% Define settings for left-hand column in which dates are printed
\newcolumntype{R}{>{\raggedleft}p{1in}}

% Define 'SectionTable' environment
\newenvironment{SectionTable}[1]{
	\renewcommand*{\arraystretch}{1.7}
	\setlength{\tabcolsep}{10pt}
	\begin{longtable}{Rp{5.2in}} & #1 \\
	}
	{
	\end{longtable}\vspace{-.3cm}
}

% Define 'SectionTableSingleSpace' environment
\newenvironment{SectionTableSingleSpace}[1]{
	\renewcommand*{\arraystretch}{1.2}
	\setlength{\tabcolsep}{10pt}
	\begin{longtable}{Rp{5.2in}} & #1 \\[0.6em]
	}
	{
\end{longtable}\vspace{-.3cm}
}

% --- Document starts here ---

\begin{document}
	

\thispagestyle{plain}  % 设置第一页的样式为 'plain'

% --- Name and contact information ---


\begin{SectionTable}{\Huge Zhijian Lai} & 
email@cornell.edu   $\;\boldsymbol{\cdot}\;$ 
www.yourwebsite.com $\;\boldsymbol{\cdot}\;$ 
314-159-2654 \newline
Citizenship: Country
\end{SectionTable}

% --- Section: Research interests ---

\begin{SectionTable}{\headingfont Research interests}
& Your favorite topic, another topic, another topic, another topic, another topic
\end{SectionTable}

% --- Section: Education ---

\begin{SectionTable}{\headingfont Education}
2019 -- Present & 
\textbf{University 1} -- City, State \newline
PhD in Subject \newline 
Mentors: Professors A, B. \textit{GPA: X.YZ}. \\

2017 -- 2019 & 
\textbf{University 2} -- City, State \newline
MA in Subject \newline 
Mentors: Professors C, D. \textit{GPA: X.YZ}. \\

2013 -- 2017 & \textbf{University 1} -- City, State \newline
BA in Subject 1, minor in Subject 2\newline 
Mentors: Professors E, F. \textit{GPA: X.YZ}. \\

% --- Un-comment the next few lines if you want to include some courses you've taken ---

& \textbf{Selected coursework}
\begin{itemize}[itemsep=0pt, leftmargin=*]
\item \textit{Statistics}: Asymptotic statistics, Mathematical statistics, Functional data analysis, High-dimensional statistics, Information theory
\item \textit{Mathematics}: Measure theory, Functional analysis, Measure-theoretic probability with martingales
\end{itemize}

\end{SectionTable}

% --- Section: Awards, scholarships, etc ---

\begin{SectionTableSingleSpace}{\headingfont Honors and scholarships}
2021 & 
Award 1 (Organization that gave you the award)\newline
\textit{Maybe this award needs a short description}. \\

2020 &
Award 2 (Organization that gave you the award; \href{https://en.wikibooks.org/wiki/LaTeX/Hyperlinks}{link if you want}) \\

2019 &
Award 3 (Organization that gave you the award) \\

2018 &
Award 4 (Organization that gave you the award) \\

2017 &
Award 5 (Organization that gave you the award)
\end{SectionTableSingleSpace}

% --- Section: Publications ---

\begin{SectionTable}{\headingfont Publications} 
2021 & 
\textbf{Title of your most recent research paper} \newline
First author, second author, third author, fourth author. \newline
\textit{Journal of something or the other}. \\

2020 & 
\textbf{Title of your second most recent research paper} \newline
First author, second author, third author, fourth author. \newline
\textit{Journal of something or the other}. \\

2020 & 
\textbf{Title of your third most recent research paper} \newline
First author, second author, third author, fourth author. \newline
\textit{Journal of something or the other}. \\

2019 & 
\textbf{Title of your fourth most recent research paper} \newline
First author, second author, third author, fourth author. \newline
\textit{Journal of something or the other}.
\end{SectionTable}

% --- Section: Research experience ---

\begin{SectionTable}{\headingfont Research experience}
Month Year -- Present &
\textbf{Title of project or lab where research was conducted} \newline
Mentors: Professor A (University). \newline
Description of your work. Summary of findings available \href{https://en.wikibooks.org/wiki/LaTeX/Hyperlinks}{here}. Sed dolor lacus, imperdiet non, ornare non, commodo eu, neque. Integer pretium semper justo. \\

Month Year -- Month Year &
\textbf{Title of project or lab where research was conducted} \newline
Mentors: Professor B (University). \newline
Description of your work. Summary of findings available \href{https://en.wikibooks.org/wiki/LaTeX/Hyperlinks}{here}. Sed dolor lacus, imperdiet non, ornare non, commodo eu, neque. Integer pretium semper justo. \\
\end{SectionTable}

% --- Section: Teaching experience ---

\begin{SectionTable}{\headingfont Teaching experience}
Fall 2020 & 
\textbf{Teaching assistant, STAT 123: Course name here (University)} \newline
Topics and description of your responsibilities. Aliquam volutpat est vel massa. Sed dolor lacus, imperdiet non, ornare non, commodo eu, neque. \newline
\textit{Average student rating: X/5.} \\

Spring 2020 & 
\textbf{Teaching assistant, MATH 234: Course name here (University)} \newline
Topics and description of your responsibilities. Aliquam volutpat est vel massa. Sed dolor lacus, imperdiet non, ornare non, commodo eu, neque. \newline
\textit{Average student rating: X/5.}
\end{SectionTable}

% --- Section: Industry experience ---

\begin{SectionTable}{\headingfont Industry experience}
Summer 2020 &
\textbf{Name of company (Title of job or internship)} -- City, State \newline
Description of your responsibilities. Integer pretium semper justo. Proin risus. Nullam id quam. Nam neque. Phasellus at purus et lib ero lacinia dictum.  \\

Summer 2019 &
\textbf{Name of company (Title of job or internship)} -- City, State \newline
Description of your responsibilities. Integer pretium semper justo. Proin risus. Nullam id quam. Nam neque. Phasellus at purus et lib ero lacinia dictum.  \\

Summer 2018 &
\textbf{Name of company (Title of job or internship)} -- City, State \newline
Description of your responsibilities. Integer pretium semper justo. Proin risus. Nullam id quam. Nam neque. Phasellus at purus et lib ero lacinia dictum.  \\
\end{SectionTable}

% --- Section: Talks and tutorials ---

\begin{SectionTable}{\headingfont Talks and tutorials}
Month Year &
Title of your most recent presentation \newline
\textit{Name of conference, workshop, seminar, etc., or a description} \\

Month Year &
Title of your second most recent presentation \newline
\textit{Name of conference, workshop, seminar, etc., or a description} \\

Month Year &
Title of your third most recent presentation \newline
\textit{Name of conference, workshop, seminar, etc., or a description} \\
\end{SectionTable}

% --- Section: Mentorship and service ---

\begin{SectionTable}{\headingfont Mentorship and service}
Month Year -- Present &
\textbf{Title of organization you are in (Name of your role)} \newline
Description of your responsibilities. Integer pretium semper justo. Proin risus. Nullam id quam. Nam neque. Phasellus at purus et lib ero lacinia dictum. \\

Month Year -- Month Year &
\textbf{Title of organization you were in (Name of your role)} \newline
Description of your responsibilities. Integer pretium semper justo. Proin risus. Nullam id quam. Nam neque. Phasellus at purus et lib ero lacinia dictum. \\
\end{SectionTable}

% --- Section: Professional society memberships ---

\begin{SectionTable}{\headingfont Professional memberships}
Year -- Present &
Name of professional society \newline
\textit{Short description or conferences you attended.} \\

Year -- Present &
Name of professional society \newline
\textit{Short description or conferences you attended.} \\
\end{SectionTable}

\begin{SectionTable}{\headingfont Technical skills}
& \textbf{Programming languages} \newline
Proficient in: language 1, language 2, language 3 \newline
Familiar with: language 4, language 5 \\

& \textbf{Software} \newline
\LaTeX, Git, another piece of software \\

& \textbf{Languages} \newline
English (fluent), Another language (advanced)
\end{SectionTable}

% --- Section: Other interests/hobbies ---

\begin{SectionTable}{\headingfont Other interests}
& Some of your hobbies, etc.
\end{SectionTable}

% --- End of CV! ---

\label{LastPage}  % 在文档的最后添加一个标签,以便于引用总页数。

\end{document}





